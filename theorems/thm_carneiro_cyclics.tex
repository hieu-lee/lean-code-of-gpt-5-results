\begin{theorem}[Carneiro analog for cyclics (disproves Conj.~50 of \cite{Cohen2025})]\label{thm:carneiro_cyclics}
For all $n$ with $c_n>3$, $c_{n+1}-c_n<\sqrt{c_n\,\log c_n}$.
\end{theorem}

\begin{proof}
We disprove the statement by an explicit counterexample.

Compute cyclic integers up to 11. By definition (Szele's criterion \cite{Szele1947}), $m$ is cyclic iff $\gcd(m,\varphi(m))=1$.
- $8$: $\varphi(8)=4$, so $\gcd(8,4)=4\ne1$ (not cyclic).
- $9$: $\varphi(9)=6$, so $\gcd(9,6)=3\ne1$ (not cyclic).
- $10$: $\varphi(10)=4$, so $\gcd(10,4)=2\ne1$ (not cyclic).
- $11$: prime, hence $\gcd(11,\varphi(11))=\gcd(11,10)=1$ (cyclic).

Thus the consecutive cyclic integers around $7$ are $7$ and $11$, so $c_{n}=7$ and $c_{n+1}=11$ for some $n$, and
$$
 c_{n+1}-c_n=11-7=4.
$$
Now note that $\log 7<2$ (since $e^2\approx7.389>7$), hence
$$
 \sqrt{7\log 7}<\sqrt{14}<4.
$$
Therefore, for $c_n=7>3$ we have $c_{n+1}-c_n=4\not<\sqrt{c_n\,\log c_n}$, contradicting the conjectured inequality.

Hence the conjecture is false. 
\end{proof}

