\begin{theorem}[Full closed forms for \seqnum{A248982} (resolves Fried's conjecture)]\label{thm:fib-full}
Let $(a_n)_{n\ge1}$ be the lexicographically least sequence of pairwise distinct positive integers such that each running average $A_n:=\tfrac{1}{n}\sum_{i=1}^n a_i$ is a Fibonacci number. Then for every $m\ge5$,
\[
 A_{2m-1}=F_{m+2},\quad A_{2m}=A_{2m+1}=F_{m+3},
\]
and
\[
 a_{2m}=2m\,F_{m+3}-(2m-1)\,F_{m+2}=F_{m+2}+2mF_{m+1},\qquad
 a_{2m+1}=F_{m+3}.
\]
In particular, for all $n\ge10$,
\[
 a_n=
 \begin{cases}
 n\,F_{\,\frac n2+3}-(n-1)\,F_{\,\frac n2+2}, & n\text{ even},\\[4pt]
 F_{\,\frac{n+1}{2}+2}, & n\text{ odd}.
 \end{cases}
\]
\end{theorem}

\begin{proof}
Let $\mathcal F=\{F_k:k\ge0\}$ and for the greedy sequence $(a_n)_{n\ge1}$ put $S_n=\sum_{i=1}^n a_i$ and $\overline a_n=S_n/n$. By hypothesis $\overline a_n\in\mathcal F$ for all $n$, so write $\overline a_n=A_n\in\mathcal F$. Then for $n\ge2$,
$$
 a_n=S_n-S_{n-1}=nA_n-(n-1)A_{n-1}.
$$
Since $A\mapsto nA-(n-1)A_{n-1}$ is strictly increasing, the lexicographically least sequence subject to positivity and pairwise distinctness is produced greedily by choosing at each step the smallest admissible $A_n\in\mathcal F$ that yields a positive new $a_n$ distinct from $\{a_1,\dots,a_{n-1}\}$.

A direct greedy computation gives
$$(a_1,\dots,a_{11})=(1,3,2,6,13,5,26,8,53,93,21),$$
$$(A_1,\dots,A_{11})=(1,2,2,3,5,5,8,8,13,21,21).$$

We prove by induction on $m\ge5$ the assertions
\[
\begin{aligned}
A_{2m-1}\,&=F_{m+2},\\
A_{2m}\,&=A_{2m+1}=F_{m+3},\\
a_{2m}\,&=2m\,F_{m+3}-(2m-1)\,F_{m+2}\\
\,&=F_{m+2}+2mF_{m+1},\\
a_{2m+1}\,&=F_{m+3}.
\end{aligned}
\]
This yields the stated closed form for all $n\ge10$ by writing $n=2m$ or $n=2m+1$.

Base step ($m=5$). From the recorded values $A_9=F_7$, $A_{10}=A_{11}=F_8$, hence
$$a_{10}=10F_8-9F_7=F_7+10F_6=93,\qquad a_{11}=F_8=21,$$
which matches $(\ast)$ for $m=5$.

Inductive step. Assume $(\ast)$ holds for some $m\ge5$. Then
$$
S_{2m-1}=(2m-1)F_{m+2},\qquad S_{2m}=2mF_{m+3},\qquad S_{2m+1}=(2m+1)F_{m+3}.
$$
We determine $A_{2m+2}$ and $A_{2m+3}$ greedily.

1) Even step $2m+2$. If $A_{2m+2}=A_{2m+1}=F_{m+3}$, then
$$a_{2m+2}=(2m+2)F_{m+3}-(2m+1)F_{m+3}=F_{m+3},$$
repeating $a_{2m+1}$. Any choice $A_{2m+2}<F_{m+3}$ gives
$$a_{2m+2}\le(2m+2)F_{m+2}-(2m+1)F_{m+3}=F_{m+2}-(2m+1)F_{m+1}<0.$$
Thus the smallest admissible choice is $A_{2m+2}=F_{m+4}$, yielding
$$
 a_{2m+2}=(2m+2)F_{m+4}-(2m+1)F_{m+3}=F_{m+3}+2(m+1)F_{m+2}.
$$
This strictly exceeds $F_{m+3}$, so it cannot collide with any earlier odd Fibonacci value $F_8,\dots,F_{m+3}$. It is also distinct from earlier even values: using the inductive formula for $a_{2m}$,
$$
 a_{2(m+1)}-a_{2m}=[F_{m+3}+2(m+1)F_{m+2}]-[F_{m+2}+2mF_{m+1}]
 =F_{m+1}+2F_{m+2}+2mF_m>0,
$$
so among even indices the values are strictly increasing, hence $a_{2m+2}>a_{2m}\ge a_{10}=93$. Finally, the only earlier odd, non-Fibonacci values are $a_7=26$ and $a_9=53$ from the initial segment; since $m\ge5$ implies $F_{m+2}\ge F_7=13$, $F_{m+3}\ge F_8=21$, and $m+1\ge6$, we have
$$
 a_{2m+2}=F_{m+3}+2(m+1)F_{m+2}\ge 21+12\cdot13=177>53,
$$
so $a_{2m+2}$ is new. This matches the even-index formula in $(\ast)$ with $m\mapsto m+1$.

2) Odd step $2m+3$. Any $A_{2m+3}<F_{m+4}$ forces
$$a_{2m+3}\le (2m+3)F_{m+3}-(2m+2)F_{m+4}=F_{m+3}-(2m+2)F_{m+2}<0,$$
so the minimal admissible choice is $A_{2m+3}=F_{m+4}$, giving $a_{2m+3}=F_{m+4}$. This value is new among odd indices because the previous odd terms are the strictly increasing Fibonacci numbers $F_8,\dots,F_{m+3}$. It remains to show that no earlier even term equals $F_{m+4}$.

Fix any earlier even index $2r\le 2m$. If $r<5$, then $a_{2r}\in\{3,6,5,8\}<F_9=34\le F_{m+4}$. If $r\ge5$, then by the inductive formula $a_{2r}=F_{r+2}+2rF_{r+1}$. Suppose toward a contradiction that $a_{2r}=F_{m+4}$. Put $s=m+2-r\ge0$. By the addition formula,
$$
F_{m+4}=F_{r+s+2}=F_{r+2}F_{s+1}+F_{r+1}F_s.
$$
Hence
$$
F_{r+2}(F_{s+1}-1)=F_{r+1}(2r-F_s).
$$
Since $\gcd(F_{r+2},F_{r+1})=1$, there exists $t\in\mathbb Z$ with
$$
F_{s+1}-1=tF_{r+1},\qquad 2r-F_s=tF_{r+2}.
$$
Because $F_{s+1}\ge1$ and $F_{r+1}\ge F_6=8$, we must have $t\ge0$. For $r\ge5$ one has $F_{r+2}>2r$ (indeed $F_7-10=3>0$, and $(F_{(r+1)+2}-2(r+1))-(F_{r+2}-2r)=F_{r+1}-2\ge6$, so the difference increases). If $t\ge1$, then $2r-F_s=tF_{r+2}\ge F_{r+2}>2r$, impossible since the left-hand side is $\le 2r$. Thus $t=0$, whence $F_{s+1}=1$ and $F_s=2r$. But $F_{s+1}=1$ forces $s\in\{0,1\}$, so $F_s\in\{0,1\}$, contradicting $2r\ge10$. Therefore no even term equals $F_{m+4}$, and $a_{2m+3}=F_{m+4}$ is new. This establishes the odd-index formula in $(\ast)$ with $m\mapsto m+1$ and also $A_{2m+2}=A_{2m+3}=F_{m+4}$.

By induction, $(\ast)$ holds for all $m\ge5$. Writing $n=2m$ or $n=2m+1$ gives, for all $n\ge10$,
$$
 a_n=
 \begin{cases}
 n\,F_{\,\frac n2+3}-(n-1)\,F_{\,\frac n2+2}, & n\text{ even},\\[4pt]
 F_{\,\frac{n+1}{2}+2}, & n\text{ odd}.
 \end{cases}
$$
Finally, the running averages satisfy $A_{2m}=A_{2m+1}=F_{m+3}\in\mathcal F$ by construction, the values $(a_n)$ are pairwise distinct because odd-index terms are the strictly increasing $F_8,F_9,\dots$ and even-index terms are strictly increasing and never equal to any of those odd values, and at each step $A_n$ is the smallest admissible choice; hence the sequence is lexicographically least among all sequences with Fibonacci running averages and distinct terms. 
\end{proof}
