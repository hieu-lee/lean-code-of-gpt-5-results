\begin{thebibliography}{99}

\bibitem{BGT1989}
N.~H. Bingham, C.~M. Goldie, and J.~L. Teugels, Regular Variation, Cambridge Univ. Press, 1989.

\bibitem{Cohen2025}
J.~E. Cohen, Conjectures about primes and cyclic numbers, J. Integer Seq. 28 (2025), Art. 25.4.7. Available at \url{https://cs.uwaterloo.ca/journals/JIS/VOL28/Cohen/cohen41.pdf}.

\bibitem{deBruijn1970}
N.~G. de Bruijn, Asymptotic Methods in Analysis, 3rd ed., North-Holland, 1970.

\bibitem{Szele1947}
T. Szele, \emph{\"Uber die endlichen Ordnungszahlen, zu denen nur eine Gruppe geh\"ort}, Comment. Math. Helv. 20 (1947), 265--267.

\bibitem{Nagura1952}
J. Nagura, On the interval containing at least one prime number, Proc. Japan Acad. 28 (1952), 177--181.

\bibitem{Pollack2022}
P. Pollack, Numbers which are orders only of cyclic groups, Proc. Amer. Math. Soc. 150 (2022), 515--524.

\bibitem{OEIS-A003277}
OEIS Foundation, Entry \seqnum{A003277}: Numbers $n$ such that every group of order $n$ is cyclic, \url{https://oeis.org/A003277}.

\bibitem{OEIS-A248982}
OEIS Foundation, Entry \seqnum{A248982}: Lexicographically earliest sequence of distinct positive integers whose averages are Fibonacci numbers, \url{https://oeis.org/A248982}.

\bibitem{Fried2025}
S. Fried, Proofs of Several Conjectures From the OEIS, J. Integer Seq. 28 (2025), Article 25.4.3. Available at \url{https://cs.uwaterloo.ca/journals/JIS/VOL28/Fried/fried15.pdf}.

\bibitem{Apostol1976}
T. M. Apostol, Introduction to Analytic Number Theory, Springer, 1976.

\bibitem{MV2007}
H. L. Montgomery and R. C. Vaughan, Multiplicative Number Theory I. Classical Theory, Cambridge Univ. Press, 2007.

\bibitem{IK2004}
H. Iwaniec and E. Kowalski, Analytic Number Theory, AMS Colloquium Publications, Vol. 53, 2004.

\bibitem{HalRich1974}
H. Halberstam and H.-E. Richert, Sieve Methods, Academic Press, 1974.

\bibitem{Greaves2001}
G. Greaves, Sieves in Number Theory, Springer, 2001.

\bibitem{FI2010}
J. B. Friedlander and H. Iwaniec, Opera de Cribro, AMS Colloquium Publications, Vol. 57, 2010.

\bibitem{Iwaniec1978}
H. Iwaniec, Almost-primes represented by quadratic polynomials, Invent. Math. 47 (1978), 171--188.

\bibitem{BHP2001}
R. C. Baker, G. Harman, and J. Pintz, The difference between consecutive primes. II, Proc. London Math. Soc. (3) 83 (2001), no. 3, 532--562.

\bibitem{Tenenbaum2015}
G. Tenenbaum, Introduction to Analytic and Probabilistic Number Theory, 3rd ed., AMS, 2015.

\bibitem{RosserSchoenfeld1962}
J. B. Rosser and L. Schoenfeld, Approximate formulas for some functions of prime numbers, Illinois J. Math. 6 (1962), 64--94.

\end{thebibliography}
